\documentclass[11pt]{article}


\usepackage{inputenc}[utf8x]
\usepackage[russian]{babel}




\usepackage{pgfplots}
\pgfplotsset{compat=1.16}
\pgfplotsset{major grid style={dashed,black}}
\pgfplotsset{minor grid style={solid,black}}
\usepackage{float}



\title{Матан - ДЗ 1}
\author{Артем Глубшев}
\date{}
\begin{document}

\maketitle

\section{Задача 1}
Задача 1. 
\begin{itemize}
\item
 Объясните, что означают перечисленных ниже утверждения (скажите другими
словами и как можно короче!).

\begin{itemize}
\item $\forall  \varepsilon  > 0 \exists N \in N \exists n > N : |a_n - a| < \varepsilon $ В последоваетльности есть точка $a$ или у последовательности есть подпоследовательность, стремящася к $a$. Действительно, рассмотрим $\varepsilon$-окрестность точки $a$. По выражению выше, в нее попадет хотя бы один член. Далее рассотрим меньшую окрестность, где $\varepsilon$ равно расстоянию от $a$ до этого члена. В нее снова попадет точка. И т.д. Ясно, что наличие в последовательности точки $a_n=a$ тоже удовлетворяет условию.



Ответ: $$a_k=a$$
\begin{center}OR
\end{center}$$lim_{n\to \infty}b_{n} \in a_n =a$$
\item $\forall  \varepsilon  > 0 \exists N \in N \forall   n > N : |a_n - a| < \varepsilon $

Точка $a$ - предел $a_n$
\item $\exists \varepsilon  > 0 \exists N \in N \forall  n > N : |a_n - a| < \varepsilon $

По утверждению, после второго члена все члены попадают в $\varepsilon$-окрестность $a$. Про $\varepsilon$ при этом ничего не сказано. Это значит, что последовательность ограничена сверху и снизу после 2-го члена, а значит - и после первого, ведь он-то точно конечен.
 
 Ответ: $a_n$ - ограниченная последовательность
 
\item $\exists \varepsilon  > 0 \forall  N \in N \exists n > N : |a_n - a| \ge \varepsilon $

Это отрицание утв.1. Предел последовательности не равен $a$.

\item $\forall  \varepsilon  > 0 \forall  N \in N \forall  n > N : |a_n - a| < \varepsilon $

Все члены последовательности со второго равны $a$. Действительно, $\forall  N \in N \forall  n > N$ эквивалентно $\forall  n>1$, то есть все члены последовательности кроме первого сколь угодно близки к $a$, то етсь равны $a$.

Ответ: Все члены последовательности со второго равны $a$.

\end{itemize}
\item  Найдите среди этих высказываний пары таких, что одно является отрицанием другого.
Ответ: 
$\forall  \varepsilon  > 0 \exists N \in N \forall  n > N : |a_n - a| < \varepsilon $
$\exists \varepsilon  > 0 \forall  N \in N \exists n > N : |a_n - a| \ge \varepsilon $
\end{itemize}

\section{Задача 2}
Построить пример последовательности an, для которой
\begin{itemize}
\item все члены последовательности $a_n$
\item множество всех натуральных чисел
\end{itemize}
являются ее предельными точками (пункты задачи независимы друг от друга).

В первом случае - пределом любой константы является она сама, т.е. любая точка постоянной последовательности - ее предельная точка. Можно взять $a_n=1$, можно навернуть что-то страшное из корней и постоянных.

Во втором случае - нам нужна такая последвательность, чтобы из нее выделялись подпоследовательности, стремящиеся к каждому из нат. чисел. Пусть это будут константный последовательности. Можно сделать последовательность вида $a_n=1,2,3,4,5,6,7 \ldots 1,2,3,4$, которая содержит бесконечность подпоследовательностей вида $b_n=b,b,b,b$.

Доказать, что число членов в ней равно числу членов в нормальной последовательности, несложно - просто строим биекцию между ней и рациональными числами - числитель - число, знаменатель - какой раз оно встретилось. А даллее, биекцию между рациональными и натуральными числами мы уже строили. А число членов в нормальной последовательности очевидно равно числу натуральных чисел.  

Ответ:
\begin{itemize}
\item $a_n = \frac{e^{\pi}-\phi}{\sqrt{RSA-2048}}$
\item $a_n = {1,2,3,4, \ldots,1,2,3,4...}$
\end{itemize}

\section{Задача 3}
Найдите пределы (если они существуют) или докажите расходимость:
\begin{itemize}
\item $lim_{n \to \infty}\frac{1}{n^3+1} (n^3 + 3n cos n) =lim_{n \to \infty}\frac{1}{1+\frac{1}{n^3}} (1 + \frac{3 cos n}{n^2}) = 1 $

Делим все на $n^3$ и пренебрегаем бесконечно малыми членами.

\item $lim_{n \to \infty}\frac{sin 2^n}{n^3+1} =0$

По лемме о двух милиционерах,$$lim_{n \to \infty}\frac{-1}{n^3+1} \le lim_{n \to \infty}\frac{sin 2^n}{n^3+1}\le lim_{n \to \infty}\frac{1}{n^3+1}$$

Ясно,что левый и правый стремятся к нулю, тогда и средний член стремится к нулю

\item $$lim_{n \to \infty}13^{-n}(n \ln n - n) = 0$$
$$lim_{n \to \infty}13^{-n}(n \ln n - n) = \frac{n(\ln n -1)}{13^n}$$
$$\frac{n}{13^n}\le\frac{n(\ln n -1)}{13^n}\le\frac{n^2}{13^n}$$
Рассмотрим <<дискретную производную>> левой и правой части:
$$\lim_{n \to \infty} \frac{\frac{n+1}{13^{n+1}}}{\frac{n}{13^n}} = \lim_{n \to \infty}\frac{(n+1)13^n}{n 13^{n+1}} = \frac1{13}$$
 
$$\lim_{n \to \infty} \frac{\frac{(n+1)^2}{13^{n+1}}}{\frac{n^2}{13^n}} = \lim_{n \to \infty}\frac{(n+1)^2 13^n}{n^2 13^{n+1}} = \frac1{13^2}$$

Ясно, что и левая и правая часть убывают - у них каждый следующий член меньше предыдущего, причем в пределе это соотношение стремится к $\frac1{13}$. Итак, два жандарма идут к нулю - и нашей функии просто некуда деться.


\item $$a_n =\frac{\sqrt{n}\cdot 3^n}{n!} = 0$$
Рассмотрим $\frac{a_{n+1}}{a_n}$
$$\frac{\frac{\sqrt{n+1}\cdot 3^{n+1}}{(n+1)!}}{\frac{\sqrt{n}\cdot 3^n}{n!}} = \frac{\sqrt{n+1}\cdot 3}{\sqrt{n}\cdot(n+1)}$$
Ясно, что на бесконечности это соотшношение стремится к $\frac3{n+1}$, что явно менье 1, что есть последовательность монотонно убывает. 

Также очевидно, что любой из членов последовательности больше нуля - ни одна из частей функции не станет меньше нуля никогда. Значит, имеем функцию,монотонно убывающую и ограниченную нулем снизу. 

\end{itemize}


\section{Задача 4} 
Петя шел из дома в школу. На полпути он решил, что плохо себя чувствует, и
повернул обратно. На полпути к дому ему стало лучше, и он повернул в школу. На полпути к iколе он решил, что все-таки нездоров, и повернул к дому. Но на полпути к дому снова повернул
к школе, и т.д. Куда придет Петя, если будет так идти?

База индукции: приняв положение школы за 1000, а дома за 0, рассчитаем первые итерации:
$$500.0
;250.0
;625.0
;312.5
;656.25
\ldots
;666.6660
;333.3330
;666.6665
\ldots
$$

Шаг Индукции: итак, допустим, что расстояние от точки $333,(3)$ до текущей позиции Пети равно $\varepsilon$, то есть он в точке $333.(3)+\varepsilon$. $\epsilon$ я откладываю к центру исходя из базы индукции.

Тогда его расстояние от школы равно $666,(6)-\varepsilon$. Он проходит половину этого расстояния, и его приходит в точку $333.(3) + \frac{666.(6)}{2} + \frac{\varepsilon }{2}$. затем он рвзворачивается и проходит половинку расстояния до дома, равного $666,(6)-\frac{\varepsilon}{2}$ и оказывается в точке $333.(3)+\frac{\varepsilon}{4}$

Итак, я показал, что с каждой итерацией расстояние между точками $333,(3)$ и $666,(6)$ на четных и нечетных тиерациях соответственно, уменьшается.

\section{Задача 6}

С помощью любого математического пакета (например, можно воспользоваться онлайн-ресурсом Geogebra https://www.geogebra.org/graphing) постройте эскиз графика функции $$f(x) = \frac 1x \sin \frac 1x$$
(приложите эскиз графика к работе). Укажите какой-нибудь интервал, на котором функция $f(x)$ обратима и какой-нибудь интервал, на котором функция $f(x)$ не обратима
(ответ объясните).


\begin{tikzpicture}
\begin{axis}[
	xlabel = {$x$},
	ylabel = {$y$},
	xmin=-0.5,
    xmax=0.5,
    ymin=-10,
    ymax=10,
    samples=10000,
	xtick = {-1,-0.8,...,1},
    ytick={-10,-8,...,10},
    grid=both,
    ]
\addplot[black] {1/x*sin(1/x)};
\end{axis} 
\end{tikzpicture}

Функция обратима только если монотонна. Участок $[-0.4;0.4]$ - необратим, участок $[0.2;0.4]$ обратим

\section{Задача 7}
Доказать равенство, используя определение сходимости по Коши:

$$lim_{x\to 0}sin(6\pi x) = 0$$

Пусть мы взяли $\varepsilon >0$. Необходимо доказать, что существует такое $\delta(\varepsilon)$, что $\forall  x < \delta |sin 6\pi x -0|<\varepsilon$

Найдем $\delta(\varepsilon)$. 
$$|sin(6\pi \delta)| < \varepsilon$$
$$-\arcsin{\varepsilon}<6\pi\delta < \arcsin{\varepsilon}$$
Итак, мы нашли $\delta(\varepsilon)$, чем доказали предел по Коши.

\section{Задача 8}
 Функция $f(x)$ определена на всей вещественной прямой. Известно, что предел по-
следовательности значений функции $f$ в натуральных точках существует, причем $lim_{n\to \infty}f(n) = l$.

Что можно сказать о пределе функции $lim _{x\to\infty}f(x)$?

Если предел функции есть, то он равен $l$. Но его может и не быть - рассмотрим функцию 
$$f(x) = \frac{sin x}{x}; if  not x \in N$$
$$f(x) = 1; if x \in N$$


\section{Задача 9}
 Сделайте предположение, чему равен предел (если он существует), и докажите это
по определению:
\begin{itemize}
\item $$lim_{x\to4}ln(3x + 4) =\ln 16$$

Найдем $\delta(\varepsilon)$

$$ln(3(4+\delta)+4) = ln(16-\varepsilon)$$
$$ln(16)ln(3\delta) = ln(16)\frac{1}{\varepsilon}$$
$$ln(3\delta) = -ln(\varepsilon)$$
$$3\delta = e^{-ln(\varepsilon)}$$
$\delta(\varepsilon)$ существует, ЧТД

\item $lim_{x\to3}\frac5{|x-3|} = \infty$
Докажем, что $$\forall  n>N \exists \varepsilon :\frac5{|x-3|}<\varepsilon \Leftrightarrow f(x)>N$$
$$\frac{5}{3+\epsilon-3} = \frac 5{\varepsilon} > N$$
$$\varepsilon = \frac 5N$$
Мы нашли $\varepsilon(N)$, доказав тем самым предел по Коши
















\end{itemize}
\end{document}





















