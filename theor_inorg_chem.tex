\documentclass[11pt]{article}
\usepackage[utf8]{inputenc}
\usepackage[russian]{babel}
\usepackage[margin=5mm]{geometry}

\title{\textbf{Теоретическая Неорганическая Химия}\\ {\normalsize (3 модуль)}}
\author{1 курс ФХ HИУ ВШЭ}
\date{23 октября 2019}
\usepackage{graphicx}
\begin{document}
\begin{titlepage}
\maketitle
\end{titlepage}
\tableofcontents
\section*{Леция 3-4. Изменение электронных оболочек атомов. Ионные соединения}

\center{Правило Клечковского}

Заполнение происходит в порядке возрастания суммы квантовых чисел $n+l$

Основное и возбужденное состояния - соответственно состояния с наименьшей энергией, и все другие состояния. При переходе из возбужденного состояния в основное выделяется энергия, например в виде излучения. Возможен и обратный вариант.

Это широко используется для анализа и изучения вещества - единственный недеструктивный метод изучения.

В пределах одного уровня орбитали неодинаковы по энергии. Различие по энергии между уровнями и подуровнями тем ниже, чем выше эти (под)уровни ($\Delta(2s-3s)>\Delta(3s-4s)$)

Спаренные электроны невыгодны по двум причинам - закон Кулона против сближения двух отрицательно заряженных частиц, и энергия, затрачиваемая на изменение спина.

Спин можно считать моментом импульса электрона, так как его движение наиболее близко к вращательному.

Изучать состояния электронов можно по частотам излучения и соответствующим переходам.
\subsection*{Экранирование электронов}
Орбитали "заслоняются" нижележащими орбиталями, энергия этого состояния повысится, связь электрона с ядром ослабевает, и т.д.

Эффективный заряд ядра - считается по правилу Слейтера. Эффективный заряд для, например, $3s NA$ - около 2.5. Приводит это к тому, что электрон от него оторвать несложно.

$$E_{eff} = Z - S$$

$S$ - константа экранирования.
Считаем ее так:
\begin{itemize}
\item Пишем конфигурацию
(1s) (2s, 2p) (3s, 3p) (3d) (4s, 4p) (4d) (4f) (5s, 5p) (5d)…
Расположите электроны согласно правилу Клечковского.
\item Любые электроны, расположенные справа от интересующего вас электрона, не оказывают влияния на константу экранирования.
\item Константа экранирования для каждой группы рассчитывается как сумма следующих составляющих:
\item Все остальные электроны, находящиеся в одной группе с интересующим нас электроном, экранируют 0,35 единиц заряда ядра. Исключение составляет 1s-группа, где один электрон считается только за 0,30.
\item В случая группы, относящейся к [s, p] типу, берут 0,85 единиц на каждый электрон (n-1) оболочки и 1,00 единицу на каждый электрон (n-2) и следующих оболочек.
\item В случая группы, относящейся к [d] или [f] типу, берут 1,00 единицу на каждый электрон, находящийся слева от этой орбитали.
\end{itemize}

Из-за проникающей способности возникает много чего интересного, например, влияние ядра на $5s$ больше чем на $4d$.

Все эти закономерности приводят по сути к периодическому закону, открытому Менделеевым за четыре десятка лет до появления вот этого физического обоснования.

\section*{Периодичность изменения свойств элементов}
Свойства:
\begin{itemize}
\item Атомный и ионный радиусы - проблемные величины, так как есть правило неопределенности. Определить радиус нельзя, но можно определить радиус участка, в котором вероятность нахождения электрона равна 0.9 или 0.8 или чему-то еще
\item Энергия ионизации и сродства к электрону
Кстати, энергия ионизации всегда положительна, а вот сродство к электрону может иметь любой знак
\item Электроотрицательность
\end{itemize}

По периоду вправо энергия ионизации растет, ЭО тоже растет из-за увеличения заряда ядра и заряда оболочки, при том что число оболочек не меняется.

По группе вниз энергия ионизации и ЭО падают, так как становится больше оболочек. 

Правда, у энергии ионизации есть немонотонность, в отличие от всех остальных характеристик, меняется менее линейно. Так, ЩМ из-за хорошей проникающей способности $s$-орбитали, в отличие от инертных газов, слабо меняют энергию ионизации по группе.

Сродство к электрону - энергия, выделяемая или поглощаемая при присоединении электрона. По группе вниз падает, по периоду вправо растет, но это тренды в целом, а так все немонотонно и зависит от устойчивости получаемой частицы.

Так, ЩЗМ принимают электроны сильно менее охотно, чем ЩМ, так как ЩМ принимает электрон на тот подуровень, что уже начал заполняться, а ЩЗМ - на новый, более высокий по энергии.

Электроотрицательность - веичина синтетическая, считается по разным формулам, отражает способность оттягивать электроны на себя.
\begin{itemize}
\item Поллинг
$$\chi = k*(\left(X-Y\right)_meas - \left(X-Y\right)_expected)$$
\item Малликен
$$\chi = \frac 12 \left(|F| + |I|\right)$$
\item Олред-Рохов
$$\chi = k\frac{Z_eff}{r^2} + c$$
\end{itemize}
Все три модели дают схожие цифры, так что использовать можно любую. Константы $k$ и $c$ - используют для "подгона" друг к другу.




\end{document}

