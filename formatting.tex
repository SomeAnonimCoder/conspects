
\documentclass[11pt]{article}
\usepackage{color}
\usepackage{listings}
\usepackage{inputenc}[utf-8]
\usepackage[russian]{babel}

\lstset{ %
language=C,                 % выбор языка для подсветки (здесь это С)
basicstyle=\small\sffamily, % размер и начертание шрифта для подсветки кода
numbers=left,               % где поставить нумерацию строк (слева\справа)
numberstyle=\tiny,           % размер шрифта для номеров строк
stepnumber=1,                   % размер шага между двумя номерами строк
numbersep=5pt,                % как далеко отстоят номера строк от подсвечиваемого кода
backgroundcolor=\color{white}, % цвет фона подсветки - используем \usepackage{color}
showspaces=false,            % показывать или нет пробелы специальными отступами
showstringspaces=false,      % показывать или нет пробелы в строках
showtabs=false,             % показывать или нет табуляцию в строках
frame=single,              % рисовать рамку вокруг кода
tabsize=2,                 % размер табуляции по умолчанию равен 2 пробелам
captionpos=t,              % позиция заголовка вверху [t] или внизу [b] 
breaklines=true,           % автоматически переносить строки (да\нет)
breakatwhitespace=false, % переносить строки только если есть пробел
escapeinside={\%*}{*)}   % если нужно добавить комментарии в коде
}


\title{\textbf{Способы форматирования информации}\\ (надеюсь, я правильно понял, чего вы хотели)}
\author{Артем Глубшев}
\date{}
\begin{document}

\maketitle

\section{Форматирование текстом}
Существует довольно много методов форматирования, основанных на читсо текстовых файлах в том или ином формате. Пример тому - \LaTeX, который мспользован для создания вот этого вот pdf-файла. 
Итак, примерный список таких форматов с примерами:
\begin{itemize}
\item{\TeX и его логический потомок \LaTeX. Пример кода:
\begin{lstlisting}[label=LaTeX,caption=LaTeX]
\documentclass[12pt]{article}
\usepackage{ucs}
\usepackage[utf8x]{inputenc} 
\usepackage[russian]{babel}  
\title{\LaTeX}
\date{}
\author{}

\begin{document}
  \begin{eqnarray}
    E &=& mc^2\\
    m &=& \frac{m_0}{\sqrt{1-\frac{v^2}{c^2}}}
  \end{eqnarray}
\end{document}
\end{lstlisting}
  \begin{eqnarray}
    E &=& mc^2\\
    m &=& \frac{m_0}{\sqrt{1-\frac{v^2}{c^2}}}
  \end{eqnarray}
}

\item{HTML+CSS
\begin{lstlisting}[label=HTML,caption=HTML]
<div class="center">
    <div id="messages" class="message-container scroll"></div>
    <div class="edit-area">
        <textarea id="message" class="message-edit"></textarea>
        <div class="send-wrap">
            <img id="send" class="send-icon" src="send.jpg">
        </div>
    </div>
</div>

\end{lstlisting}
}


\item{HTML+CSS
\begin{lstlisting}[label=CSS,caption=CSS]
* {
font-family: fantasy;
border-radius: 10px ;
}

/*HEADER*/
.head{
    margin:10px;
    padding:10px;
    width: 95%;
    background: linear-gradient(to right, #99aabb, #556677);
    margin:auto;
    text-align: center;
    padding: 20px;
    font-size:36px;
    height:30%;
}

/*MAIN CONTENT*/
body{
    background: linear-gradient(to top, #708090, #8090a0);
}

/*MENU AND CENTER*/
.center{
    background: linear-gradient(to right, #99aabb, #556677);
    margin:10px;
    width:75%;
    height:50%;
    float:right;
    border:3px groove ghostwhite;
}

.left-menu{
    background: linear-gradient(to right, #99aabb, #556677);
    margin:10px;
    padding:10px;
    width:20%;
    border:3px groove ghostwhite;
    float:left;
}

/*MESSAGES AND ONE MESSAGE*/
.message-container{
    margin:10px;
    padding:10px;
    overflow-x:auto;
    height:400px;
    background: linear-gradient(to right, #bbccdd, #99aabb);
}


\end{lstlisting}
}


\item{Еще миллион других форматов, которыми я не владею: markdown, javadoc, etc
}

\end{itemize}

\section{Нетекстовые форматы}
\begin{itemize}
\item{pdf. Изначально вообще не рассматривался как формат для электронного просмотра, а был форматом для печати, но жизнь распорядилась иначе}
\item{Форматы электронных книг - тысячи их. FB2, mobi, djvu, etc}
\item{.doc(x) и родственные ему .odf и другие офисные форматы - основаны на xml, если гугл не врет}
\item{в целом, никто не запретит мне сохранять текст в картинках. Хоть для таких людей и приготовлен отдельный котел в аду, всякие jp(e)g, bmp,svg - почему бы и не да?}
\end{itemize}

\end{document}
