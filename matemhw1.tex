\documentclass[11pt]{article}


\usepackage{inputenc}[utf8]
\usepackage[russian]{babel}

\title{Матан - ДЗ 1}
\author{Артем Глубшев}
\date{}
\begin{document}

\maketitle

\section{Задача 0}
$$N = 5, M=7, L=8$$
\section{Задача 1}

Итак, имеем множество отрезков. Тот факт, что для максимальной суммы длин <<кусочков>> множества необходимо максимизировать размер отрезков и минимизимировать расстояние между ними, предлагаю принять очевидным.

Тогда начинаем от левого края, откладываем максимально возможный по условию отрезок отрезок с длинной равной $0.1-\varepsilon$. Ясно, что больше нельзя - нарушим условия, а меньше можно, но нас же просят максимальный! Затем отступ, минимально возможный по условию - $0.1-\varepsilon$. Ясно, что сделав отступы меньше получим точки, запрещенные условием, сделав отрезки длиннее - тоже. Повторив пятикратно такое отложение, получим 5 отрезков длинной $0.1 -	\varepsilon$, то есть сумма длин отрезков равна $0.5-5\varepsilon$. Устремив $\varepsilon$ к нулю, получим ответ: сумма длин отрезков сколь угодно близка к $\frac12$ снизу, но не может быть равна или больше. 

\textbf{Ответ: Да, доказательство в предыдущем абзаце}

\section{Задача 2}

Методом рандома(не шучу, правда сгенерировал и рандомно выбрал) получено: 89, 97, 47. Теперь к решению задачи.

Ясно, что на 89 делятся $floor(\frac{1000}{89})$ чисел. Аналогично для 97 и 47. Но данный ответ имеет проблему: если чило делится и на 97, и на 47, я посчитаю его дважды, что есть нехорошо. Исправим же это. Сперва определим, сколько чисел делятся на 97 и 47 одновременно. Мне повезло, 0, ведь даже 47*89 больше 1000, а уж замена одного из множителей на 97 дает еще большее число. Таким образом, мне не надо заморачиаться с этим всем, ну и хорошо.

$$
\frac{1000}{47} = 21.27659574468085
$$$$
\frac{1000}{89} = 11.235955056179776
$$$$
\frac{1000}{97} = 10.309278350515465
$$$$
21+11+10=42
$$

Полным перебором, занявшим аж целых 0.245 секунды получен тот же ответ.

\textbf{Ответ: 42. Кажется, мой компьютер шутит.}

\section{Задача 3}

Итак, мы знаем что $$x+\frac{1}{x} \in N$$. Это утверждение - наша \textbf{база индукции} - доказуемое верно для степени равной 1.

Допустим, что для чисел вплоть до $n$ уже доказано,что $$x^n + \frac{1}{x^n} \in N$$

Тогда $$x^{n+1} + \frac{1}{x^{n+1}} = (x^n +\frac{1}{x^n})(x +\frac{1}{x})-(x^{n-1} - \frac{1}{x^{n-1}})$$

Все скобки в выражении справа от равно целые, а сумма, разность и умножение целых чисел дают целые.

\textbf{Доказано}

\section{Задача 4}

Я люблю использовать подходы, скоращающие путь к результату, так что сейчас будет простое неиндукционное доказательство, сделать кторое можно в пятом классе, если вспомнить \textbf{Формулу Бине}
$$F_n = \frac{\Phi^n  - (-\Phi)^{-n}}{2\Phi-1}$$

Здесь $\Phi = 1.61833... = \frac{1+\sqrt{5}}{2}$ - соотношение серебрянного сечения, "красивого" деления отрезка, такого, что большее к меньшему относится как целое к большему.

Так, начинаем доказывать. 

$$F_{n+1} * F_{n-1} = \frac{\Phi^{n+1} - (-\Phi)^{-n-1}}{2\Phi-1}*\frac{\Phi^{n-1} - (-\Phi)^{-n+1}}{2\Phi-1} $$

Раскрыв, получим:

$$F_{n+1} * F_{n-1} = \frac{-(-a)^{1-n} a^{n+1} + a^{2n} - (-a)^{-n-1} + (-a)^{-2n}}{2 \Phi-1}$$

Теперь то же самое для $F_n^2$:
$$F_n^2 = (\frac{\Phi^n  - (-\Phi)^{-n}}{2\Phi-1})^2 = a^{2n} - 2a^n(-a)^{-n} + (-a)^2n$$

Вычтем

$$F_{n+1} * F_{n-1} - F_n^2n = -(-a)^{1-n}a^{n+1} + 2a^n(-a)^{-n} - (-a)^{-1-n}a^{n-1}$$

Вынесем $a^n$:

$$F_{n+1} * F_{n-1} - F_n^2n = a^n(-a(-a)^{1-n} + 2(-a) ^{-n} - (-a)^{-1-n}a^{-1})$$
Упростим:
$$F_{n+1} * F_{n-1} - F_n^2n = a^n((-a)^{2-n} + 2(-a) ^{-n} + (-a)^{-2-n})$$
Свернем в нечто похожее на формулу суммы квадратов
$$F_{n+1} * F_{n-1} - F_n^2n = a^n({(-a)^{-n}}^{2} + 2(-a)^{-n} + {(-a)^{-n}}^{-2})$$

... Тут я расхотел набирать еще лист с гаком приведений, но в итоге после подстановки $\Phi = \frac{1+\sqrt{5}}{2} $ привелось к требуемому

$$F_{n+1} * F_{n-1} - F_n^2n = (-1)^n$$

ЧТД.

P.S. Школьник пятого класса должен быть безумно аккуратным, я с matlab'ом-то еле привел это все, а до того были три или четыре попытки на листочке, разумеется, неудачные. Ну нафиг так <<сокращать путь>>


Доказательство по индукции: 

\textbf{База} При $$n=2$$ равенство верно. 

\textbf{Шаг} Пусть доказано, что 
$$F_n^2 = F_{n-1}F_{n+1} + (-1)^n$$ 
Прибавим к обоим частям $F_nF_{n+1}$:
$$F_n^2 +F_nF_{n+1}= F_{n-1}F_{n+1} + (-1)^n + F_nF_{n+1}$$ 
Вынесем:
$$F_n(F_n+F_{n+1})= F_{n+1}(F_{n-1}+F_n) + (-1)^n$$ 
По определению, $$F_{n+2} = F_{n+1}F_n$$
$$F_nF_{n+2} = F_{n+1}^2 + (-1)^n$$

\textbf{ЧТД}

\section{Задача 5}
По лемме, доказаной ранее, корень извлекается рационально тогда и только тогда,  когда он извлекается нацело.

Кроме того, мы доказывали, что(далее $R$ - тип <<рациональное число>>, $I$ - <<иррациональное число>>, использована система записи типов haskell):
$$add:: R -> R -> R$$
$$add:: I -> I -> R$$
$$add:: I -> R -> I$$

Из всего вышесказаного следует, что $$n+\sqrt{n} \in R \Leftarrow n = k^2, k\in N$$

\textbf{Ответ: Для $n=k^2, k \in N$}

\section{Задача 6}

\subsection{}
15 мест для герани, 5 для роз. Отсюда $15!$ и $5!$ перестановок там и там, и соответственно, всего вариантов  $$N=15!*5!$$ 
\subsection{}
Сводится к первой, введением дополнительной степени свободы - номера горшка с геранью, после которого вы впихнем розы. 
$$N = 15!*5!*16$$
\subsection{}
герань ставим $15!$ вариантами. Теперь для первой розы 16 вариантов впихивания, для второй 15, 14, 13, и 12.
$$N=15!\frac{16!}{11!}$$
\textbf{Ответ: a. $15!*5!$ b.$15!*5!*16$ c.$15!*\frac{16!}{11!}$}

\section{Задача 7}

Вы, вероятно, не хотели услышать этот вариант ответа, но сколь угодно много. Берем три дощечки длиной $N$, делаем из них треугольник. Потом шесть. Потом девять. Ну а что, кто заставлял использовать ровно три дощечки?

Приведу также решение исходя из дополнительного условия о том. что дощечек всякий раз ровно три. С моими числами проблем с треугольниками быть не может, действительно, минимальная сторона у меня 6, максимальная 9, $6+6>9$, нарушить неравенство треугольника я не могу. Тогда рассмторим три случая: все стороны разные, две совпадают, и все равны. 

В перовм случае имеем 4 варианта, так как можем не включить лишь одну длинну дощечек. 

Во втором случае имеем 12 вариантов - 4 для парной стороны и 3 для непарной, предполагая что мы выбираем первой парную, иначе наоборот, но не суть.

В третьем случае вариантов снова 4, ведь какая разница, включить одну или все кроме одной. 

\textbf{Ответ: 20 вариантов}

\section{Задача 8}

\subsection{}

Никакой. Сумма степеней не равна M+L+N, такого члена не будет

\subsection{}
Степень у меня 20, искомый член - $a^5b^7c^8$. Вспоминаем формулу бинома Ньютона, которую грех не расширить на полиномы, и вот вам - полиномиальная формула.
$$(a_1+a_2+..+a_k)^n = \sum \frac{n!}{r_1!r_2!..r_k!}a_1^{r1}a_2^{r2}..a_k^{rk}$$

Нетрудно видеть, что коэффициент равен $\frac{20!}{5!8!7!}$

\textbf{Ответ:a. 0; b. $\frac{20!}{5!8!7!}$;}

\section{Задача 9}
\subsection{}
Почти полная копия задачи с семинара про квадратный город, решается в три строчки:

40 шагов до цели - следовательно, $40!$ вариантов подразумевая что все шаги различны;

10, 14 и 16 эквивалентны - значит делим на их факториалы;

Легко видеть, что получается:

$$N=\frac{40!}{10!14!16!}$$ 

\subsection{}

Нет, не зависит, из любого варианта можно перейти в любой, используя отражение кубика, которое не меняет взаимоотношений его частей

\section{Задача 10}
L*M*N = 280

Очевидно, что группа обязана содержать четное число К., так как мальчиков и девочек поровну и пилить их нельзя. Итак, возможны компании из 2, 4, 6 ... 280*2 К. Для компании из 2n зверей мальчиков можно выбрать $\frac{280!}{n!(280-n)!}$, девочек тоже. Тогда для компании из 2n имеем $$(\frac{280!}{n!(280-n)!})^2$$ вариантов

Итак, общее число вариантов: 
$$\sum_{i=1}^{280}\frac{280!}{n!(280-n)!}$$
Существует теория, что это приводится к чему попроще, но мне не удалось.

\textbf{Ответ:$$\sum_{i=1}^{280}\frac{280!}{n!(280-n)!}$$}

\section{Задача 11}

Пишем в ряд $N$ единиц - их сумма очевидно $N$
ставим или нет между ними столбики - $N-1$ мест, так как с краев нельзя, $0 \notin N$. Итак, имеем $2^{N-1}$ способов поставить столбики, но есть одно <<но>> - нельзя по условию не поставить столбиков вообще, так как представить $N$ как $N$ нельзя, $N$ не меньше $N$. Отсюда минус один. Получили, что N имеет $2^{N-1}-1$ способов представления, \textbf{чтд}

\end{document}





















